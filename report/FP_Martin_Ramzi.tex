\documentclass[a4paper,12pt]{article}

\usepackage[utf8]{inputenc} % allow utf-8 input
\usepackage{hyperref}       % hyperlinks
\usepackage{url}            % simple URL typesetting
\usepackage{booktabs}       % professional-quality tables
\usepackage{amsfonts}       % blackboard math symbols
\usepackage{nicefrac}       % compact symbols for 1/2, etc.
\usepackage{microtype}      % microtypography
\usepackage{graphicx}       % include graphics
\usepackage{subcaption}     % subfigures
\usepackage{float}          % placement of floats
\usepackage{fancyhdr}       % head notes and foot notes



\graphicspath{ {images/} }


\title{Joint representations for images and text}
\author{Louis Martin and Zaccharie Ramzi}


\begin{document}

\maketitle

% TODO: put this as a subtitle
Multimodal retrieval: image-to-image search, tag-to-image search,
and image-to-tag search.

\begin{abstract}
\end{abstract}

\section{Introduction}

% TODO: cite with "Hotelling et al. (1936)" instead of "[1]"
Canonical Correlation Analysis (CCA) was first proposed by \cite{originalcca}.
The objective is to find a common basis between two correlated spaces (two datasets), such that the correletion between the vectors in the new basis is maximized.
KCCA uses a kernel method to map the initial spaces to higher dimensional spaces before applying CCA. This usually increases the computational power of CCA.

% TODO: use this citation
\cite{hardooncca}


\begin{thebibliography}{7}

\bibitem{originalcca}
  Hotelling, H.
  (1936). 
  \emph{Relations between two sets of variables.}
  Biometrika, 28, 312–377.
  % http://cbio.mines-paristech.fr/~jvert/svn/bibli/local/Hotelling1936Relation.pdf

\bibitem{hardooncca}
  Hardoon, D., Szedmak, S., and Shawe-Taylor, J.
  (2004).
  \emph{Canonical correlation analysis; an overview with application to learning methods.}
  Neural Computation.
  % http://citeseerx.ist.psu.edu/viewdoc/download?doi=10.1.1.702.5978&rep=rep1&type=pdf

% TODO: Beautify following bibitems
\bibitem{normalizedcca}
  \emph{Normalized CCA},
  \url{http://slazebni.cs.illinois.edu/publications/yunchao\_cca13.pdf}
\bibitem{word2vec}
  \emph{Word2Vec},
  \url{http://code.google.com/p/word2vec/}
\bibitem{overfeat}
  \emph{Overfeat},
  \url{http://cilvr.nyu.edu/doku.php?id=software:overfeat:start}
\bibitem{vgg}
  \href{http://arxiv.org/pdf/1409.1556.pdf}{
  K. Simonyan, A. Zisserman, Very Deep Convolutional Networks for Large-Scale Image Recognition, 2014
  }
\bibitem{deepfilters}
  \href{http://www.robots.ox.ac.uk/~vgg/publications/2015/Cimpoi15/cimpoi15.pdf}{
  M. Cimpoi, S. Maji, A. Vedaldi, Deep Filter Banks for Texture Recognition and Segmentation, CVPR 2015.
  }
\end{thebibliography}

\end{document}
